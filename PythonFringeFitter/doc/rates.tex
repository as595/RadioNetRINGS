\documentclass[11pt, twoside, a4paper]{article}
\usepackage{color}
\usepackage{charter}
\usepackage{tikz-cd}
\usepackage[ruled]{algorithm2e}
\usepackage{amsmath}
\usepackage{amssymb}
\usepackage{upquote}
\usepackage{appendix}
\usepackage{graphicx}
\usepackage{listings}
\usepackage{url}
\usepackage[T1]{fontenc}  % access \textquotedbl
\usepackage{textcomp}     % access \textquotesingle
%\usepackage[backend=bibtex,natbib=true,
%            citestyle=authoryear,style=authoryear,
%            maxcitenames=2,maxbibnames=10]{biblatex}
\usepackage{authblk}
% \bibliography{papers}
\mathchardef\mhyphen="2D
\author[1]{Des Small}
\affil[1]{JIVE}
\title{Delay-rates and/or phase-rates in fringe-fitting}
\bibliographystyle{alpha}
\begin{document}\sloppy
\usetikzlibrary{decorations.pathmorphing}
\maketitle
\lstset{basicstyle=\ttfamily\footnotesize,breaklines=true}

\section{Introduction}
In fringe-fitting we compensate for instrumental and atmospheric delays
and delay-rates in the signal paths of the data reaching the correlator
by means of a correction applied \emph{after} correlation.

The delay term (more properly, the \emph{group delay}\cite{petrov11})
is computed as the gradient of the phase across a frequency band.  It
has units of seconds; it corresponds to a physical delay; it is not a
fertile source of confusion or controversy. The second term usually
included in fringe-fitting is related to the observed rotation of the
correlation phasor as a function of time, and this term is more subtle.
We show the relation between the two popular perspectives on this term
(``phase-rate'' and ``delay-rate'') and discuss their implications for
calculating, storing and applying calibration terms in the
fringe-fitting process.

\section{Phase rates and/or delay rates}
The first sign of trouble came when we compared plots of delay-rates
from AIPS FRING routine (see for example \cite{mao15}), which use by
default a y-axis scaled in millihertz with the textual form of the
results from the same routine, which are given in a dimensionless (or s/s) form.

Appendix C of the AIPS manual (p. C-42) defines the ``residual phase''
stored in solution and calibration tables as a result of fringe fitting
as follows
\begin{equation}\label{eq:aips:rate}
  \begin{split}
    \Phi_{i,c}(\nu, t) = &2\pi\left\{\nu_{i,0} - \nu_{0,0}\right\} \mathit{MB} \\
    + &2\pi\left\{\nu_{i,c} - \nu_{i,0}\right\} \mathit{SB}_i \\
    + &2\pi \nu_{i,c} \left\{t-t_0\right\} \mathit{RATE} + \theta_{i,c} 
  \end{split}
\end{equation}
for channel $c$
of IF $i$,
where $\mathit{MB}$
is the multi-band delay, $\mathit{SB}_i$
is the single-band delay for band $i$,
$\mathit{RATE}$
is the delay rate and $\theta_{i,c}$
is the ``peculiar phase'' for an IF and channel. (We preserve the
manual's notation and terminology for convenience in
cross-referencing.) 

It is the $\mathit{RATE}$ that is stored, in dimensionless (s/s) form,
in the calibration form in the solution (``SN'') and, when applied,
calibration (``CL'') tables. As Equation~\ref{eq:aips:rate} shows, this
rate is multiplied separately with the frequency of each spectral
channel when it the correction is applied.

It turns out that the values are stored in the orthodox
dimensionless form (s/s) in the solution tables, but for presentation
they are scaled up by the frequency of the carrier wave, since the
period of those waves determine the ``ambiguity time'' of the results.

During the actual fringe fitting process itself, however, the
natural\footnote{Sadly, English lacks a convenient equivalent for ``voor de
  hand liggend''.} 
form of the same quantity is that of a phase-rate, as in the
expression given by Schwab and Cotton\cite{schwab83} for baseline
visibilities $\widetilde{V}_{ij}$, of the form
\begin{equation}\label{eq:phaserate}
  \begin{split}
  \widetilde{V}_{ij}(t_k, v_l) &\simeq a_i a_j \mathcal{V}_{ij}(t_0, \nu_0) 
  \exp
  \left\{i
    \left[(\psi_i-\psi_j)(t_0, \nu_0) \right]
  \right\} \\
  &\times\exp
  \left[i\left(
      \left.
        \frac{\partial (\psi_i-\psi_j + \phi_{ij})}{\partial t}
      \right|_{t_0, \nu_0} (t_k-t_0) \right.\right.
        \\
      &+ \left.\left.\left.
        \frac{\partial (\psi_i-\psi_j + \phi_{ij})}{\partial \nu}
      \right|_{t_0, \nu_0} (\nu_l-\nu_0)
      \right] \right),
  \end{split}
\end{equation}
with a true visibility of $\mathcal{V}_{ij}$,
antenna gains of $a_i$,
a model term of $\phi_{ij}$,
and a grid of times, $t_k$,
and frequencies, $\nu_l$ with origin at $t_0, \nu_0$.

This is the form used for the 2-dimensional Fourier transform; the
baseline delay-rate term in this form is 
\[
\left.\frac{\partial (\psi_i-\psi_j)}{\partial t} \right|_{t_0,
    \nu_0} (t_k-t_0),
\]
corresponding to a station-based rate which can be equated with the
AIPS form as follows,
\begin{equation}
  \frac{\partial \psi_i}{\partial t} (t_k - t_0) \simeq
  2\pi \nu_{i,c} \left\{t-t_0\right\} \mathit{RATE}.
\end{equation}

It is interesting to note that the peak of the two-dimensional Fourier
transform defines a \emph{single} phase rate across the band, as do
the global non-linear refinements subsequently applied by AIPS FRING procedure;
converting this to a delay-rate in seconds per second requires the
choice of a reference frequency to divide by, and when it is scaled
back up by the channel frequencies it will be slightly (and arguably
artificially) different across the band. But our bands are narrow and
our goal is simple: we want to do what AIPS does, because that's what
the VLBI community wants. It is suspected that AIPS uses the lower
edge of the frequency band as its reference frequency (for both upper
and lower subbands, since it doesn't make such a distinction).

For many astronomical purposes, a crucial concept is the
\emph{ambiguity time}, defined as the time for a full cycle of the
phasor, i.e., the reciprocal of the sky frequency of the wave. When the
delay rate multiplied by the time of a scan approaches the ambiguity
time or, equivalently, when the phase has wrapped around by $2\pi$.
For C-band at $5\,\mathrm{GHz}$ a (high) rate of
$1\,\mathrm{ps}/\mathrm{s}$ corresponds to $5\,\mathrm{mHz}$, so that
phase will wrap in $200\,\mathrm{s}$.

\section{Conclusion}
For some purposes (notably, for comparison with CALC's model
coefficients and checking station clocks) it is convenient to think of
delay rates in seconds per second; in other cases it is handier to
think in terms of radians per second. AIPS converts freely between the
two, and we should follow its example wherever possible down to the
finest detail.


\bibliography{fringe}
\end{document}

\begin{figure}
  \centering
  \includegraphics[height=8cm]{afigure.eps}
  \caption{A caption}
  \label{fig:finalEngine}
\end{figure}




