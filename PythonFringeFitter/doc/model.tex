\documentclass[11pt, twoside, a4paper]{article}
\usepackage{color}
\usepackage{charter}
\usepackage[ruled]{algorithm2e}
\usepackage{amsmath}
\usepackage{amsfonts}
\usepackage{url}
\usepackage{upquote}
\usepackage{appendix}
\usepackage{graphicx}
\usepackage{listings}
\usepackage[T1]{fontenc}  % access \textquotedbl
\usepackage{textcomp}     % access \textquotesingle
\usepackage[backend=bibtex,natbib=true,
            citestyle=authoryear,style=authoryear,
            maxcitenames=2,maxbibnames=10]{biblatex}
\usepackage{authblk}
\bibliography{fringe}

\mathchardef\mhyphen="2D
\author[1]{Des Small}
\affil[1]{JIVE}
\title{Fringe fitting in Casa -- deriving, storing and applying coefficients}
\begin{document}\sloppy
\maketitle
\lstset{basicstyle=\ttfamily\footnotesize,breaklines=true}

\section{Introduction}
This report briefly sets out the definition of parameters used in the
fringe fitting routines we are adding to Casa, and sketches how we
calculate store and apply the values of the parameters. It is primarily
intended to record the design decisions made about reference
frequencies and times used to index the parameters; it will be seen
that we follow the example of AIPS wherever possible. It is explicitly
not intended as a tutorial on fringe fitting or a detailed account of
our methods.

Note that we do not currently implement a facility for using a source
model of a phase-reference source; this will need to be implemented in
a later release.

\section{A model of station phases}
We assume that the observed behaviour of phase on each baseline can be
modeled as the combination of antenna-based effects.

For an antenna indexed with $l$,
and visibility data on a grid of $f_i$
in frequency and $t_j$
in time, we model the phase of the visibilities as the first-order
Taylor series around a reference point $(f_\mathrm{ref}, t_\mathrm{ref})$.

\begin{equation}
  \phi_l(f, t) \approx \phi_{\mathrm{ref}, l} + 
  \frac{\partial \phi_l}{\partial f} \delta f +
  \frac{\partial \phi_l}{\partial t} \delta t
\end{equation}
where
\begin{align}\label{eq:pardef}
   \phi_{\mathrm{ref},l} &= \phi_l(f_{\mathrm{ref}}, t_{\mathrm{ref}}) \\
   \delta f &=  f - f_{\mathrm{ref}} \\
   \delta t &= t -t_{\mathrm{ref}}
\end{align}
For compactness of notation we replace the partial derivatives with the
group delay, $\tau_l$ and phase delay rate, defined as
$r_l$ as 
\begin{align}
  \tau_l &= \frac{\partial \phi_l}{\partial f} \\
  r_l &= \frac{\partial \phi_l}{\partial t},
\end{align}
where we use the terminology, but not the symbols, of \cite{petrov11}.

In fringe fitting we do not consider the magnitudes of the
visibilities, only the phases, and we model the relative phase of a
baseline between two antennas with indices $l$
and $m$ as the difference between their modeled phases of the two antennas.

Accordingly we define the complex model unit
visibilities, $\hat{\mathcal{V}}$, by
\begin{equation}\label{eq:model}
\hat{\mathcal{V}}_{lm} \approx \exp\left(\mathfrak{j} \left\{
    \Delta \phi_{\mathrm{ref}, lm} + 
    2\pi \Delta \tau_{lm} f + 
    2\pi \Delta r_{lm} t
    \right\}
    \right),
\end{equation}
where the hat denotes the normalisation of the visibilites; we use
Fraktur $\mathfrak{j}$
for the square root of minus one to keep Roman letters free for
subscripts; and we introduce the shorthand notation for differences
that

\begin{align}\label{eq:diffabbr}
  \Delta \phi_{\mathrm{ref}, lm} &=   \Delta \phi_{\mathrm{ref}, l} -   \Delta \phi_{\mathrm{ref}, m} \\
  \Delta \tau_{lm}  &=   \Delta \tau_{l} - \Delta \tau_{m} \\
  \Delta r_{lm}  &=   \Delta r_{l}  -   \Delta r_{m} 
\end{align}

We have assumed that visibilities are provided on a grid of frequency
points and times defined by the spectral channels and integration time
(or ``accumulation time'') of the correlator for each subband
(``spectal window'' in Casa, ``IF'' in AIPS), so that the comparison
between the model and the data is made on a uniform grid of frequency
$f_i$
and time $t_j$,
although some data may of course be flagged. The interval of time used
in a single fringe fit is referred to as a ``solution interval''; it
may not be -- and often isn't -- the same as the length of a ``scan''
(defined as a continuous observation of a single source).

In the case of multi-band fringe-fitting we currently assume that all
spectral windows share the same frequency spacing and are almost
contiguous.  In practice, upper and lower sidebands from the same
frequency down conversion will share a DC edge, and the Nyquist channel
will be omitted, so a little care is needed filling the grid.

Our model of phase on baseline $l$-$m$, either for a single subband or
for the full set of bands, at the point $(f_i, t_j)$ is thus
\begin{equation}\label{eq:discmodel}
\hat{\mathcal{V}}_{lm, ijXS} = \exp\left(\mathfrak{j} \left\{
      \Delta \phi_{\mathrm{ref}, lm} +     
      2\pi \Delta \tau_{lm, ij} (f_{i} - f_{ref}) + 
      2\pi \Delta r_{lm, ij} (t_{j} - t_{ref})
 \right\}
\right).
\end{equation}

\section{Computing, storing and applying the model}
\subsection{Computing}

We want to solve for the parameters $\phi_{\mathrm{ref}, l}$,
$\tau_l$
and $r_l$
for each antenna, but we only have data about differents between
parameters from baselines formed from pairs of antennas.  To fix an origin
we designate one antenna as the \emph{reference antenna}, for which all
the parameters are defined as zero.

The full details of the solution process are beyond the scope of this
note, but in brief: we use the baselines to the reference antenna to
produce preliminary estimates of the antenna parameters via a
two-dimensional Fourier transform, and then we feed those estimates to
a global least-squares solver for refinement. 

We choose the centre time and frequency point for our reference values
($t_{\mathrm{ref}})$, $f_{\mathrm{ref}}$ in Equation~\ref{eq:pardef}),
where we restrict the time range for 

\subsection{Storing and applying}
The calculated parameters then have to be stored in a Casa calibration
table using the \texttt{gencal} routine, and can then be applied by the
\texttt{applycal} routine. For this purpose, all the routines involved
have to agree on the reference time and frequency for which values are
reported. In particular, since the \texttt{applycal} routine is being
extended to support delay rates as part of the current project to
implement VLBI support in Casa, we need to \emph{define} these
reference values.

Following the established practice of AIPS -- the \emph{de facto}
standard for VLBI data processing -- we use the centre frequency in an
frequency band as the reference frequency, regardless of whether the
frequency band is labelled as an upper or lower sideband, and ignoring
the \texttt{REF\_FREQUENCY} column in the Measurement Set
\texttt{SPECTRAL\_WINDOW} table, which is intended to record a fact
about the observation. More than one definition of ``centre'' is
possible; we propose to define the centre frequency as the
$\frac{n}{2}$th
frequency point in a band of $n$
frequency points, where the first (or rather, zeroth) point is the
lowest frequency in the band.

When calculating the model parameters we divide the data in the time
dimension into what, again following AIPS, we call ``solution
intervals''; it is good practice to make this interval shorter
than the coherence time of the atmosphere. The AIPS convention, which
we once again adopt, is to use the center of the solution interval as
the reference time for the parameters.

\end{document}

% \begin{figure}
%   \centering
%   \includegraphics[height=8cm]{afigure.eps}
%   \caption{A caption}
%   \label{fig:finalEngine}
% \end{figure}

