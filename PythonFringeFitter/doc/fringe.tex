\documentclass[11pt, twoside, a4paper]{article}
\usepackage{color}
\usepackage{charter}
\usepackage{tikz-cd}
\usepackage[ruled]{algorithm2e}
\usepackage{amsmath}
\usepackage{amssymb}
\usepackage{upquote}
\usepackage{appendix}
\usepackage{graphicx}
\usepackage{listings}
\usepackage{url}
\usepackage[T1]{fontenc}  % access \textquotedbl
\usepackage{textcomp}     % access \textquotesingle
%\usepackage[backend=bibtex,natbib=true,
%            citestyle=authoryear,style=authoryear,
%            maxcitenames=2,maxbibnames=10]{biblatex}
\usepackage{authblk}
% \bibliography{papers}
\mathchardef\mhyphen="2D
\author[1]{Des Small}
\affil[1]{JIVE}
\title{What we talk about when we talk about manual phase-cal correction}
\bibliographystyle{alpha}
\begin{document}\sloppy
\usetikzlibrary{decorations.pathmorphing}
\maketitle
\lstset{basicstyle=\ttfamily\footnotesize,breaklines=true}

\section{Introduction}

Mao\cite{mao15} has argued. \cite{diamond}

\section{Phase rates and/or delay rates}
Plots of delay-rates from AIPS (see for example \cite{mao15}) use axes
scaled in millihertz. The values are stored in the orthodox
dimensionless form ($s/s$) in the solution tables, but for presentation
they are scaled up by the frequency of the carrier wave, since the
period of those waves determine the ``ambiguity time'' of the results.

For C-band at $5\,\mathrm{GHz}$ a (high) rate of
$1\,\mathrm{ps}/\mathrm{s}$ corresponds to $5\,\mathrm{MHz}$, so that
phase will wrap in $200\,\mathrm{s}$.

Appendix C of the AIPS manual (p. C-42) gives the ``residual phase''
stored in solution and calibration tables as a result of fringe fitting
as
\begin{equation}
  \begin{split}
    \Phi_{i,c}(\nu, t) = &2\pi\left\{\nu_{i,0} - \nu_{0,0}\right\} \mathit{MB} \\
    + &2\pi\left\{\nu_{i,c} - \nu_{i,0}\right\} \mathit{SB}_i \\
    + &2\pi \nu_{i,c} \left\{t-t_0\right\} \mathit{RATE} + \theta_{i,c} 
  \end{split}
\end{equation}
for channel $c$ of IF $i$, where $\mathit{MB}$ is the multi-band array, $\mathit{SB}_i$ is
the single-band delay for band $i$, $\mathit{RATE}$ is the delay rate
and $\theta_{i,c}$ is the ``peculiar phase'' for an IF and channel.

Meanwhile, Schwab and Cotton express visibilities $\widetilde{V}_{ij}$ as 
\begin{equation*}
  \begin{split}
  \widetilde{V}_{ij}(t_k, v_l) &\simeq a_i a_j \mathcal{V}_{ij}(t_0, \nu_0) 
  \exp
  \left\{i
    \left[(\psi_i-\psi_j)(t_0, \nu_0) \right]
  \right\} \\
  &\times\exp
  \left[i\left(
      \left.
        \frac{\partial (\psi_i-\psi_j + \phi_{ij})}{\partial t}
      \right|_{t_0, \nu_0} (t_k-t_0) \right.\right.
        \\
      &+ \left.\left.\left.
        \frac{\partial (\psi_i-\psi_j + \phi_{ij})}{\partial \nu}
      \right|_{t_0, \nu_0} (\nu_l-\nu_0)
      \right] \right),
  \end{split}
\end{equation*}
with a true visibility of $\mathcal{V}_{ij}$,
antenna gains of $a_i$,
a model term of $\phi_{ij}$,
and a grid of times, $t_k$,
and frequencies, $\nu_l$ with origin at $t_0, \nu_0$.

This is the natural form for 2-dimensional Fourier transform; the
baseline delay-rate term in this form is 
\[
\left.\frac{\partial (\psi_i-\psi_j)}{\partial t} \right|_{t_0,
    \nu_0} (t_k-t_0),
\]
Extracting a station-based rate and equating it with the AIPS form we have
\begin{equation}
  \frac{\partial \psi_i}{\partial t} (t_k - t_0) \simeq
  2\pi \nu_{i,c} \left\{t-t_0\right\} \mathit{RATE} 
\end{equation}


\begin{thebibliography}{99} 
\bibitem[Mao, 2015]{mao15}{\it Very Long Baseline Interferometry: Tutorial 6 at ERIS 2015}
\\ \url{http://www.jive.eu/~mao/ERIS2015/T6.html}

\bibitem[Diamond, 1995]{diamond} Diamond, P.J. (1995)
  {\it Chapter 12: VLBI Data Reduction in Practice} in {\it Zenus et  al. (Eds) 1995}

\bibitem[Cotton, 1995]{Cotton95} Cotton, W.D. (1995) {\it
  Fringe-Fitting} in {\it Zenus et  al. (Eds)} (1995)

  
\bibitem[Zenus \emph{et al.} (Eds), 1995]{zensus} Zensus, J.A., Diamond, P.J.,
  Napier, P.J. {\it Very Long Baseline Interferometry and the VLBA},
  ASP Conference Series, Vol. 82, 1995
\end{thebibliography}
%% \printbibliography
\end{document}

\begin{figure}
  \centering
  \includegraphics[height=8cm]{afigure.eps}
  \caption{A caption}
  \label{fig:finalEngine}
\end{figure}




